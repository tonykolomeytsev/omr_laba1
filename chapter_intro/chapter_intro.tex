\likechapter{Постановка задачи и исходные данные}

\textbf{Цель работы} 

Управление манипуляционным роботом \textit{KUKA youBot} на основе решения обратной задачи о положениях при перемещении схвата по заданному закону.

\textbf{Ограничения}

В лабораторных рассматривается только две степени манипулятора -- остальные зафиксированы. Из этого следует, что: $ \phi_1, \phi_4 $ и $ \phi_5 $ это константы, а угол ориентации схвата в плоскости руки манипулятора: $\theta = \phi_2+\phi_3+\phi_4$. Причем, последнее звено схвата выпрямлено, находится в положении параллельном третьему звену, откуда следует что $ \phi_4=0 $.

\textbf{Исходные данные}

Длины звеньев и расстояния между их осями:
\begin{align*}
    d_1 &= 0.033 \\
    l_1 &= 0.075 \\
    l_2 &= 0.155 \\
    l_3 &= 0.135 \\
    l_4 &= 0.081 \\
    l_5 &= 0.137 \\
    l_{45} &= l_4+l_5 = 0.218 \\
    l_{345} &= l_3 + l_4 + l_5 = 0.353
\end{align*}

Схват перемещается g g g g 