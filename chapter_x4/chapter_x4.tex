\chapter{\MakeUppercase{ОПРЕДЕЛЕНИЕ ЗАКОНОВ ИЗМЕНЕНИЯ ОБОБЩЕННЫХ КООРДИНАТ НА ПРОГРАММНОМ ДВИЖЕНИИ}}

Используя метод верзоров и кинематических винтов, составим уравнения кинематики шестизвенного манипулятора. После чего численно решим эти уравнения с использованием ранее полученных начальных условий.
$$ G_{i, i+1} =\begin{bmatrix}
    \Gamma_{i, i+1} & 0 \\
    r_{O_1, O_{i, i+1}} \Gamma_{i, i+1} & \Gamma_{i, i+1}
\end{bmatrix}, U_A = \begin{bmatrix}
    \omega \\
    V_A
\end{bmatrix} $$

\noindent где $ U_A $ - кинематический винт, $ G $ - верзор. Верзоры вычисляются следующим образом: 
\begin{align*}
    & G_1 = G_{01} \\
    & G_2 = G_1 \cdot G_{12} \\
    & \vdots \\
    & G_i = G_{i-1} \cdot G_{i, i+1}
\end{align*}

\noindent матрица $ r_{O_1, O_{i, i+1}} $ вычисляется как показано ниже:
$$ r_{O_1, O_{i, i+1}} = \begin{bmatrix}
    0 & -d_{i+1} & a \sin \theta \\
    d_{i+1} & 0 & -a \cos \theta \\
    -a \sin \theta & a \cos \theta & 0
\end{bmatrix} $$