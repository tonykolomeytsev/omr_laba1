\chapter{\MakeUppercase{Построение рабочей области схвата в вертикальной плоскости}}

\section{Выражение матрицы однородного преобразования для соседних
звеньев через параметры Денавита-Хартенберга}
Используем решение прямой задачи кинематики, полученное методом однородных преобразований.

Требуется найти $ \rho_A $ - столбец однородных координат точки $ А $ в неподвижных осях.
$$ \rho_A=\begin{bmatrix}
X_A \\
Y_A \\
Z_A \\
1
\end{bmatrix} $$

Известны $ T_{i, i+1} , \rho_A^{(N)}$.

\begin{align*}
    & \rho_A^{(N-1)}=T_{N-1, N}\cdot \rho_A^{(N)} \\
    & \rho_A^{(N-2)}=T_{N-2, N-1}\cdot \rho_A^{(N-1)}=T_{N-2, N-1}\cdot T_{N-1,N} \cdot \rho_A^{(N)} \\
    & \vdots  \\
    & \rho_{A} =T_{01}\cdot T_{12}\cdot \dots \cdot T_{N-1, N} \cdot \rho_A^{(N)}
\end{align*}

Тогда матрица однородных преобразований вычисляется следующим образом: 
$$ T_{0, N}=T_{01}\cdot T_{12}\cdot \dots \cdot T_{N-1, N}, $$

\noindent причем $ T_{i, i+1} $ вычисляется, как показано ниже:
$$  T_{i, i+1} = T_{\theta_i} \cdot T_{d_i} \cdot T_{a_{i+1}} \cdot T_{\alpha_{i+1}}, $$

\noindent где $ T_\theta $ - матрица поворота вокруг оси \textit{Z}, $ T_d $ - матрица параллельного переноса вдоль оси \textit{Z}, $ T_a $ - матрица параллельного переноса вдоль оси \textit{X}, $ T_\alpha $ - матрица поворота вокруг оси \textit{X}.

\begin{align*}
    & T_\theta = \begin{bmatrix}
        \cos \theta_i & -\sin \theta_i & 0 & 0 \\
        \sin \theta_i & \cos \theta_i & 0 & 0 \\
        0 & 0 & 1 & 0 \\
        0 & 0 & 0 & 1
    \end{bmatrix}, 
    T_d = \begin{bmatrix}
        1 & 0 & 0 & 0 \\
        0 & 1 & 0 & 0 \\
        0 & 0 & 1 & d \\
        0 & 0 & 0 & 1
    \end{bmatrix}, \\
    & T_a = \begin{bmatrix}
        1 & 0 & 0 & a_{i} \\
        0 & 1 & 0 & 0 \\
        0 & 0 & 1 & 0 \\
        0 & 0 & 0 & 1
    \end{bmatrix},
    T_\alpha = \begin{bmatrix}
        1 & 0 & 0 & 0 \\
        0 & \cos \alpha_{i} & -\sin \alpha_{i} & 0 \\
        0 & \sin \alpha_{i} & \cos \alpha_{i} & 0 \\
        0 & 0 & 0 & 1
    \end{bmatrix} .
\end{align*}

\section{Рабочая область манипулятора в вертикальной плоскости}
Построение рабочей области было проведено в среде \textit{Wolfram Mathematica}. 

\begin{code}
    
\end{code}