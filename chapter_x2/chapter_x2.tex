\chapter{\MakeUppercase{ Управление по положению схвата. Численный метод Ньютона. }}

\section{Постановка задачи}
Дан вектор $ X = \begin{bmatrix}
    X_{1A}(t) \\ Z_{1A}(t)
\end{bmatrix} $, требуется найти $ q = \begin{bmatrix}
    \phi_2 \\ \phi_3
\end{bmatrix} $ методом Ньютона.

\section{Метод Ньютона}
Путем итерационных приближений находится $ q^{(k+1)}=q^{(k)}-p^{(k)} $, где $ p^{(k)} $ -- вектор полного шага.
\begin{align*}
    & J(q^{(k)}) \cdot p^{(k)} = F(q^{(k)}) \\
    & J = \frac{\partial F}{\partial q} = \begin{bmatrix}
        \frac{\partial X_{1A}}{\partial \varphi_2} & \frac{\partial X_{1A}}{\partial \varphi_3} \\ \\
        \frac{\partial Z_{1A}}{\partial \varphi_2} & \frac{\partial Z_{1A}}{\partial \varphi_3}
    \end{bmatrix}
\end{align*}

Если размерность $ J $ небольшая, то можно найти $ p^{(k)} = J^{-1}(q^{(k)}) \cdot F(q^{(k)}) $. Здесь и выше $ J $ -- это матрица Якоби. Она формируется путем дифференцирования графа координат схвата по обобщенным координатам. Граф для координат схвата:
\begin{align*}
    X_{1A}&=d_1+l_2\sin\varphi_2+l_3\sin{(\varphi_2+\varphi_3)} \\
    Z_{1A}&=l_1+l_2\cos\varphi_2+l_3\cos{(\varphi_2+\varphi_3)}
\end{align*}

При помощи найденных численно $ \varphi_1, \varphi_2 $ можно решить прямую задачу кинематики, подставив значения в граф. После чего сравнить полученную траекторию движения с программным движением.

Программа для нахождения $\varphi_2,\varphi_3$, а также графики реальных и программных значений координат представлены в Приложении 2.

Также были получены квадратичные отклонения: 
$$||X(t)-X^*||_2=0.194585$$
$$||Z(t)-Z^*||_2=0.2622$$.