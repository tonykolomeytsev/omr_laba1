\chapter{\MakeUppercase{Алгоритм управления на основе решения обратной задачи о скоростях.}}
\section{Постановка задачи}
Даны векторы $ X^* = \begin{bmatrix}
    X_{1A}^*(t) \\ Z_{1A}^*(t)
\end{bmatrix} $, $ \dot{X^*} = \begin{bmatrix}
    \dot{X}_{1A}^*(t) \\ \dot{Z}_{1A}^*(t)
\end{bmatrix} $ -- закон изменения скорости схвата.
Известно, что $ J(q)\dot{q} = \dot{X}$.

Требуется найти: $ \dot{q} = \begin{bmatrix}
    \dot{\varphi_2} \\ \dot{\varphi_3}
\end{bmatrix} $ .

\section{Алгоритм управления}

Введём вектор ошибок положения схвата: $e=X-X^*(t)=f(q)-X^*$, где $f(q)$ -- фактические координаты, $X^*$ -- требуемые координаты.

Потребуем, чтобы $e\rightarrow0$ при $t\rightarrow\infty$.

Пусть $\dot{e}=-ke,k= \begin{bmatrix}
    k1 & 0 \\ 0 & k2
\end{bmatrix} $, тогда $e\rightarrow0$ при $t\rightarrow\infty$ в силу ассимптотической устойчивости.
\begin{align*}
    \dot{e} = \dot{X}-\dot{X}^* \\
    J\dot{q}-\dot{X}^* = -ke \\
    J\dot{q} = \dot{X}^*-ke \\
    \dot{q} =J^{-1}\cdot(\dot{X}^*-ke)
\end{align*}
Для проверки результата решим прямую задачу. Получим $\varphi_2$, $\varphi_3$ из $\dot{q}$ с помощью метода Эйлера:
\begin{align*}
    \varphi_{2i} = \varphi_{2(i-1)}+0.1\dot{\varphi}_{2i} \\
    \varphi_{3i} = \varphi_{3(i-1)}+0.1\dot{\varphi}_{3i}
\end{align*}
Подставим полученные значения в граф для координат схвата:
\begin{align*}
    X_{1A}&=d_1+l_2\sin\varphi_2+l_3\sin{(\varphi_2+\varphi_3)} \\
    Z_{1A}&=l_1+l_2\cos\varphi_2+l_3\cos{(\varphi_2+\varphi_3)}
\end{align*}
Программа для нахождения $\dot{q}$, а также графики реальных и программных значений координат представлены в Приложении 3.

Также были получены квадратичные отклонения: 
$$||X(t)-X^*||_2=0.0002513$$
$$||Z(t)-Z^*||_2=0.38452$$.