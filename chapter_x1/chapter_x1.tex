\chapter{\MakeUppercase{Описание движения манипулятора с помощью параметров ДенавитаХартенберга. }}
\section{Таблица параметров Денавита-Хартенберга}
Всю схему можно разделить на основные функциональные блоки, обеспечивающие работоспособность:

\begin{table}[ht]
    \caption{Таблица параметров Денавита-Хартенберга}
    \label{table_dh}
    \centering
    \begin{tabular}{|c|c|c|c|c|c|}
    \hline Номер звена & Тип звена & $ \theta_i $ & $ d_i $ & $ d_i $ & $ \alpha_i $ \\
    \hline 1 & П & 0                            & $ d_{1}^{*} $ & $ a_1 $ & 0  \\
    \hline 2 & В & $ 0^\circ ..\: 180^\circ $   & $ d_2 $ & $ a_2 $ & 0 \\
    \hline 3 & В & $ 60^\circ ..\: 338^\circ $  & $ d_3 $ & 0 & $ -90^\circ $ \\
    \hline 4 & В & $ \pm 350^\circ $            & $ d_4 $ & 0 & $ 90^\circ $ \\
    \hline 5 & В & $ 61^\circ ..\: 299^\circ $  & $ d_5 $ & 0 & $ -90^\circ $ \\
    \hline 6 & В & $ \pm 350^\circ $            & $ d_6 $ & 0 & 0 \\
    \hline
    \end{tabular}
\end{table}

\section{Кинематические схемы манипулятора}

В приложениях А и Б приведены кинематическая схема манипулятора из спецификации, кинематическая схема с системами координат для робота KUKA KR 30 JET. Спецификация взята из каталога [1]. %  http://robotforum.ru/promyishlennyie-robotyi.html

